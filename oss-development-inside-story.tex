\documentclass[aspectratio=169,11pt,hyperref={colorlinks=true}]{beamer}
% https://github.com/zr-tex8r/BXcjkjatype/blob/master/README-ja.md
\usepackage[whole]{bxcjkjatype}
\usetheme{boxes}
\setbeamertemplate{navigation symbols}{}
\definecolor{suse}{RGB}{2, 211, 95}
\definecolor{susedark}{RGB}{13, 44, 64}
\definecolor{linkcolor}{RGB}{13, 44, 255}
\setbeamercolor{titlelike}{fg=suse}
\setbeamercolor{structure}{fg=suse}
\hypersetup{colorlinks,urlcolor=linkcolor}
\setbeamertemplate{footline}[frame number]
% Inserting graphics
\usepackage{graphicx}
% Side-by-side figures, etc
\usepackage{subfigure}
% Code snippits
\usepackage{listings}
% Color stuff
\usepackage{color}
% underline
\usepackage{soul}
% calc
\usepackage{calc}

\usepackage{amsmath}
\usepackage{tikz}
\newcommand\RBox[1]{%
  \tikz\node[draw,rounded corners,align=center,] {#1};%
}
\usepackage{hyperref}
%\usecolortheme{buzz}
%\usecolortheme{wolverine}
%\usetheme{Boadilla}
\usepackage[T1]{fontenc}
%\usepackage{fontspec}
%\usepackage[expert, deluxe]{otf}

\definecolor{mygreen}{rgb}{0,0.6,0}
\definecolor{mygray}{rgb}{0.5,0.5,0.5}
\definecolor{mymauve}{rgb}{0.58,0,0.82}


%\usepackage{CJK}
%\pdfmapline{=genshingothic@Unicode@ <genshingothic.ttf}
% bxcjkjatype
%\setgothicfont[<ID>]{<フォントファイル名>}
%\setgothicfont{/Users/foo/Library/Fonts/genshingothic.ttf}
%\setgothicfont{/Users/foo/Library/Fonts/NotoSansCJKjp-Regular.otf}
%\setgothicfont{/Users/foo/Downloads/genshingothic-20150607/GenShinGothic-P-Normal.ttf}
%\setgothicfont{/Users/foo/Downloads/genshingothic-20150607/GenShinGothic-Regular.ttf}
%\setgothicfont{hiragino.ttc}
\setgothicfont{mplus-1p-regular.ttf}
\setCJKfamilydefault{\gtdefault}
%\setCJKfamilydefault{\mcdefault}
%\CJKforce{abcdefghijklmnopqrstuvwxyzABCDEFGHIJKLMNOPQRSTUVWXYZ}


\lstset{%
  backgroundcolor=\color{white},   % choose the background color; you must add \usepackage{color} or \usepackage{xcolor}
  breakatwhitespace=false,         % sets if automatic breaks should only happen at whitespace
  breaklines=true,                 % sets automatic line breaking
  captionpos=b,                    % sets the caption-position to bottom
  commentstyle=\color{suse},  % comment style
  extendedchars=true,              % lets you use non-ASCII characters; for 8-bits encodings only, does not work with UTF-8
  keepspaces=true,                 % keeps spaces in text, useful for keeping indentation of code (possibly needs columns=flexible)
  keywordstyle=\color{blue},       % keyword style
%  otherkeywords={*,...},           % if you want to add more keywords to the set
  numbersep=5pt,                   % how far the line-numbers are from the code
  numberstyle=\tiny\color{mygray}, % the style that is used for the line-numbers
  rulecolor=\color{white},         % if not set, the frame-color may be changed on line-breaks within not-black text (e.g. comments (green here))
  showspaces=false,                % show spaces everywhere adding particular underscores; it overrides 'showstringspaces'
  showstringspaces=false,          % underline spaces within strings only
  showtabs=false,                  % show tabs within strings adding particular underscores
  stringstyle=\color{suse},   % string literal style
}

\setbeamerfont{caption}{series=\normalfont,size=\fontsize{6}{8}}
%\setbeamerfont{caption}{series=\normalfont,size=\large}
\setbeamertemplate{caption}{\raggedright\insertcaption\par}

\setlength{\abovecaptionskip}{0pt}
\setlength{\floatsep}{0pt}

%%%%%%%%%%%%%%%%%%%%%%%%%%%%%%%%%%%%%%%%%%%%%%%%%%%%%%%%%%%%%%%%%%%%%

\author[Masayuki Igawa]{%
    \texorpdfstring{%
        \begin{columns}
        \column{.7\linewidth}
            \centering
            Masayuki Igawa: \href{mailto:masayuki@igawa.io}{masayuki@igawa.io}\\
            \texttt{masayukig on
              \href{https://freenode.net/}{Freenode},
              \href{https://github.com/masayukig}{GitHub},
              \href{https://twitter.com/masayukig}{Twitter},
              \href{https://www.linkedin.com/in/masayukig/}{LinkedIn}}
        \end{columns}
        }
    {Masayuki Igawa}
}
\date{\href{https://events.opensuse.org/conferences/summitasia19/program/proposals/2762}{@Room 201, 13:45-14:30, October 6, 2019}}
\def\place#1{\def\@place{#1}}
\place{\href{https://events.opensuse.org/conferences/summitasia19}{@openSUSE.Asia Summit 2019}}

%\vspace*{30pt}
\title[oss-development-inside-story
  \hspace{4em}\insertframenumber/\inserttotalframenumber]{How to Participate/Contribute to Open Source\\
  - Open Source tooling and community in the upstream -}

\setbeamercolor{background canvas}{bg=white}
\setbeamercolor{titlelike}{fg=black}
\setbeamercolor{structure}{fg=black}
\setbeamercolor{normal text}{fg=black}

\begin{document}
\setbeamertemplate{background canvas}{\includegraphics[width=\paperwidth,height=\paperheight]{images/opensuse_base.png}}
{%
\setbeamertemplate{background canvas}{\includegraphics[width=\paperwidth,height=\paperheight]{images/opensuse_preface.png}}
\setbeamertemplate{footline}{}
\setbeamercolor{background canvas}{bg=white}
\begin{frame}[noframenumbering]
  \hypersetup{colorlinks,urlcolor=susedark}
  \setbeamercolor{author}{fg=black}
  \setbeamercolor{date}{fg=black}
  \setbeamercolor{place}{fg=black}
  \titlepage{}
  \centering
  \@place \par
  \href{https://github.com/masayukig/oss-development-inside-story}{https://github.com/masayukig/oss-development-inside-story}
  %\vspace{1em}
  \begin{flushright}
    \tiny\href{https://creativecommons.org/licenses/by/4.0/}{This work
      is licensed under a Creative Commons Attribution 4.0
      International License.}~\includegraphics[scale=0.3]{images/cc_by.png}
  \end{flushright}
\end{frame}
}

\section{Agenda}
\begin{frame}
  \frametitle{Agenda}
  \begin{enumerate}
    \item About me
    \item Today's Goal
    \item Open Source Communities
    \item OpenStack as an example
    \item True Stories
    \item How to Participate
    \item Summary
  \end{enumerate}
\end{frame}

\section{DISCLAIMER}
\begin{frame}
  \frametitle{DISCLAIMER}
  \Huge{\bf{These slides are my own opinion}}
\end{frame}

\section{About me}
\begin{frame}
  \frametitle{About me}
  \begin{itemize}
    \item Company:1998.4-2015.12 Traditional IT company in Japan,\\
      2016.1-2017.3 HPE -> 2017.3- SUSE -> 2019(\href{https://www.suse.com/c/further-independence-for-suse/}{\scriptsize{``Further Independence for SUSE''}})
      \begin{itemize}
        \item SUSE OpenStack Cloud Team
      \end{itemize}
    \item Job: Senior Software Engineer/Open Source Programmer
      \begin{itemize}
        \begin{scriptsize}
        \item \href{https://www.openstack.org/}{OpenStack}
          \href{https://wiki.openstack.org/wiki/QA}{QA} Up/Downstream development, Core Reviewer
        \item[]
          (\href{https://docs.openstack.org/developer/tempest/}{Tempest},
          \href{http://status.openstack.org/openstack-health/}{OpenStack-Health},
          \href{https://docs.openstack.org/developer/subunit2sql/}{Subunit2SQL},
          \href{https://docs.openstack.org/developer/stackviz/}{Stackviz},etc.),
          \href{https://github.com/mtreinish/stestr}{stestr}
        \item
          \href{http://stackalytics.com/?user_id=igawa&release=all&metric=all}{stackalytics.com/?user\_id=igawa},
          \href{https://github.com/masayukig}{github.com/masayukig}
        \end{scriptsize}
      \end{itemize}
    \item Books \includegraphics[scale=0.2]{images/OpenStack_Integration_book.png}~\includegraphics[scale=0.2]{images/InfraCI_book.png}
      \begin{itemize}
      \item \href{https://www.amazon.co.jp/dp/4798139785/}{\scriptsize{OpenStack
        Cloud Integration (Japanese book)}} (one of the authors)
      \item \href{https://www.amazon.co.jp/dp/4798155128/}{\scriptsize{Infra CI
        Pragmatic Guide - Ansible/GitLab (Japanese book)}} (as a reviewer)
      \end{itemize}
    \item Hobby: Jogging, Bike(BMC SLR02), Clouds(OpenStack...), Diet(Low-carb), etc.
  \end{itemize}
  \includegraphics[height=20mm]{images/my-bike.jpg}~
  \includegraphics[height=20mm]{images/server_front.jpg}~
  \includegraphics[height=20mm]{images/my-small-server.jpg}~
  \includegraphics[height=20mm]{images/my-weight.png}
\end{frame}

\section{Introduction}
\begin{frame}
  \frametitle{ }
  \Huge{\bf{Introduction}}
\end{frame}

\section{User?Developer}
\begin{frame}
  \frametitle{Are you a User/Developer...?}
  \begin{itemize}
    \item An Open Source Software isn't generated automatically
    \item Someone resolved bugs, wrote, tested, etc.
    \item Someone wrote documents, filed bugs, advertised, etc.
    \item If you're just a user, why not contribute?
  \end{itemize}
\end{frame}

\section{Today's Goal}
\begin{frame}
  \frametitle{Today's Goal}
  \begin{itemize}
    \item Development ways/tools in OpenStack/OSS communities
    \item Challenges in OSS development, but it's fun :)
  \end{itemize}
\end{frame}

\section{Open Source Communities?}
\begin{frame}
  \frametitle{Open Source Communities?}
  There are a lot of styles. All OSS projects are different.\\
  Examples: Led by
  \begin{itemize}
    \item Individual (so many, you can find a lot on github)
    \item One company (openSUSE, Ansible, Elastic Search, ..)
    \item Foundation (OpenStack, Kubernetes, Linux kernel, ..)
  \end{itemize}
  \includegraphics[height=20mm]{images/osi-keyhole.png}~
  \includegraphics[height=20mm]{images/openSUSE-logo.png}~
  \includegraphics[height=20mm]{images/openstack-logo.png}~
  \includegraphics[height=20mm]{images/kubernetes-logo.png}
  \\
  Let's take a look at the OpenStack community as an example.
\end{frame}

\section{OpenStack community as an example}
\begin{frame}
  \frametitle{ }
  \Huge{\bf{OpenStack community}}
\end{frame}

\section{What is the OpenStack?}
\begin{frame}
  \frametitle{What is the ``OpenStack''?}
  \begin{itemize}
    \item Written in Python, Open Source Cloud Operation System: Apache License 2.0
    \item One of the top three active OSS communities
    \item There are a lot of `OpenStack' projects: \href{http://governance.openstack.org/reference/projects/index.html}{63 projects(2019-10-05)}
    \item Released every 6 month: Latest version is called `Stein'
    \item Users: \scriptsize{AT\&T, AA, BBVA, Bloomberg, CERN,
      China Mobile, Gap, VEXXHOST,
      Volkswagen, WALMART, etc.. \url{https://www.openstack.org/user-stories/}}
  \end{itemize}
  \begin{center}
    \includegraphics[height=50mm]{images/openstack-simple.png}
    \includegraphics[height=50mm]{images/stein-release-logo.png}
  \end{center}
\end{frame}

\begin{frame}
  \frametitle{Actual (What is the ``OpenStack''?)}
  \begin{center}
    \includegraphics[width=1.0\textwidth]{images/openstack-arch-kilo-logical-v1.png}
  \end{center}
\end{frame}

\section{Workflow of OpenStack development}
\begin{frame}
  \frametitle{Development Workflow in OpenStack}
  \url{https://docs.openstack.org/infra/manual/developers.html}
  \begin{columns}[T]
    \begin{column}{0.5\textwidth}
      \begin{enumerate}
        \item Submit a story to StoryBoard \& discuss (Option)
        \item Make a patch and submit it to Gerrit
        \item Automated testing by Zuul
        \item Review it on Gerrit
        \item Automated testing by Zuul, and merge
      \end{enumerate}
    \end{column}
    \begin{column}{0.4\textwidth}
      \centering\includegraphics[height=45mm]{images/code_review.png}
    \end{column}
  \end{columns}
\end{frame}

\section{Communication tools}
\begin{frame}
  \frametitle{Tools \& Communication}
  \begin{itemize}
    \item IRC on Freenode: \#openstack-*(dev,nova,glance,qa,..) \url{https://freenode.net/}
    \item Mailing List: \url{http://lists.openstack.org/}
    \item StoryBoard: \url{https://storyboard.openstack.org/}
    \item Launchpad: \url{https://launchpad.net}
    \item Gerrit: \url{https://review.opendev.org/}
    \item Gitea: \url{https://opendev.org/openstack/}
    \item Zuul: \url{https://zuul.openstack.org}
    \item OpenStack-Health: \url{http://status.openstack.org/openstack-health/}
  \end{itemize}
  \centering
  \includegraphics[height=35mm]{images/freenode.png}
  \includegraphics[height=35mm]{images/openstack-ml.png}
  \includegraphics[height=35mm]{images/openstack-in-launchpad.png}
\end{frame}

\begin{frame}
  \frametitle{Project management by StoryBoard - What's ``StoryBoard''}
  \begin{itemize}
    \item Story management tool developed by the OpenStack community
    \item Built from scratch, \url{https://opendev.org/opendev/storyboard/}
    \item Archtecture: {Frontend: JS/Angular, Backend: Python/Pecan}
    \item 100\% open source (Apache2.0)
  \end{itemize}
  \centering\includegraphics[scale=0.2]{images/storyboard.png}
\end{frame}

\begin{frame}
  \frametitle{Gerrit review system - What's ``Gerrit''}
  \begin{itemize}
    \item Android development community is using it
    \item Java, 100\% open source, \url{https://www.gerritcodereview.com}
    \item No Pull Request, submit a patch and merge it into a central repo
    \item A terminal client exists > Gertty: \url{https://gertty.readthedocs.io/}
  \end{itemize}
  \centering
  \includegraphics[height=40mm]{images/gerritcodereview-com.png}
  \includegraphics[height=40mm]{images/review-openstack-org.png}
\end{frame}

\begin{frame}
  \frametitle{FYI: Gertty - Gerrit in TTY }
  \begin{itemize}
    \item Terminal client
    \item \url{https://gertty.readthedocs.io/}
    \item Demo
  \end{itemize}
  \centering
  \includegraphics[height=60mm]{images/gertty.png}
\end{frame}

\begin{frame}
  \frametitle{Gitea - Git with a cup of tea}
  \begin{itemize}
    \item A painless self-hosted Git service
    \item Golang, MIT license
    \item \url{https://gitea.io}
  \end{itemize}
  \centering
  \includegraphics[width=65mm]{images/gitea-tempest.png}
  \includegraphics[width=65mm]{images/gitea-tempest-manager.png}
\end{frame}

\begin{frame}
  \frametitle{Continuous Integration by Zuul - What's ``Zuul''}
  \begin{itemize}
    \item \url{https://zuul-ci.org/}
    \item Set a gate by automated testing - Stop Merging Broken Code    
    \item CI/CD powered by Ansible
    \item Cross-Project Dependencies can be tested
  \end{itemize}
  \centering\includegraphics[scale=0.2]{images/zuul-status.png}
\end{frame}

\begin{frame}
  \frametitle{Gate Health check by OpenStack Health}
  \begin{itemize}
    \item A dashboard for visualizing test results of OpenStack CI jobs
    \item Built from scratch, \url{https://opendev.org/openstack/openstack-health/}
    \item Archtecture: {Frontend: JS/Angular, Backend: Python/Flask}
    \item 100\% open source (Apache2.0)
  \end{itemize}
  \centering\includegraphics[width=60mm]{images/openstack-health1.png}
  \centering\includegraphics[width=60mm]{images/openstack-health2.png}
\end{frame}

\section{Inside Story}
\begin{frame}
  \frametitle{Inside Story}
  \begin{enumerate}
    \item Very long time to get merged, why?
    \item Unstable and Slow CI
    \item Communication is important
  \end{enumerate}
\end{frame}

\begin{frame}
  \frametitle{1. Very long time to get merged, why?}
  \begin{itemize}
    \item Project size ≒ Number of stakeholders
    \item Less communication
    \item Less reviewers,  Unstable and Slow CI (next slides)
  \end{itemize}
\end{frame}

\begin{frame}
  \frametitle{1. Less reviewers (cont. Why?)}
  \centering\includegraphics[height=70mm]{images/project-overview.png}
  \url{http://activity.openstack.org/dash/browser/scr.html}
\end{frame}

\begin{frame}
  \frametitle{2. Unstable and Slow CI}
  \begin{itemize}
    \item Easily rot if the code doesn't work in CI
    \item ``CI'' is not perfect
    \item Busy neighbors problem
    \item Rapid fixing is necessary - next slide
  \end{itemize}
\end{frame}

\begin{frame}
  \frametitle{2. Elastic Recheck (cont. Unstable and Slow CI)}
    Correlate a bug to a signature in a log, and detect the bug automatically
  \centering\includegraphics[width=60mm]{images/elastic-recheck.png}
  \centering\includegraphics[width=75mm]{images/elastic-recheck-source.png}
  \url{http://status.openstack.org/elastic-recheck/},
  \url{https://opendev.org/opendev/elastic-recheck/src/branch/master/queries/1737039.yaml}
\end{frame}

\begin{frame}
  \frametitle{3. Communication is important (English is hard)}
  \begin{itemize}
    \item Discussion
    \item Find a job (e.g. difficul to get good salary as a programmer basically in Japan)
    \item Fun (Take care of yourself! :)
  \end{itemize}
  \centering\includegraphics[height=60mm]{images/tweet-suse.png}
  \centering\includegraphics[height=60mm]{images/tweet-oui.png}
\end{frame}

\begin{frame}
  \frametitle{3. English as a second language (cont. Communication is important)}
  \begin{itemize}
    \item We have a document: \url{https://docs.openstack.org/doc-contrib-guide/non-native-english-speakers.html}
    \item Not only language but also culture (For example: Zensho shimasu (善処します))
  \end{itemize}
  \centering\includegraphics[height=60mm]{images/non-native-english-speakers.png}
\end{frame}

\section{How to contribute Open Source}
\begin{frame}
  \frametitle{ }
  \Huge{\bf{How to contribute to Open Source}}
\end{frame}

\begin{frame}
  \frametitle{How to contribute to Open Source}
  \begin{itemize}
    \item Find projects which you are interested in
    \item Find online/offline mentor(s) and friends
    \item Understand culture of communities
    \item Not only code, using/testing, submit bugs/issues, documentation, translation, etc.
    \item As a study, hobby or job?
    \item Find events: \href{https://docs.openstack.org/upstream-training/\#upcoming-trainings}{OpenStack Upstream training},
     \href{https://summerofcode.withgoogle.com/}{GSoC},
     \href{https://hacktoberfest.digitalocean.com/}{Hacktoberfest}
    \item etc.
  \end{itemize}
  \centering\includegraphics[height=45mm]{images/hacktoberfest.png}
\end{frame}

\section{Conclusion}
\begin{frame}
  \frametitle{Conclusion}
  \begin{itemize}
    \item Gerrit, StoryBoard, Zuul, OSS can make our delvelopment efficiently
    \item Communication is important, and English too
    \item Contributing to OSS is fun! and tough sometimes. Why not participate in it?
  \end{itemize}
  Appendix
  \begin{itemize}
      \item Slides: \url{https://github.com/masayukig/oss-development-inside-story/blob/master/oss-development-inside-story.pdf}
      \item Contact info: \texttt{masayukig on
        \href{https://freenode.net/}{Freenode},
        \href{https://github.com/masayukig}{GitHub},
        \href{https://twitter.com/masayukig}{Twitter},
        \href{https://www.linkedin.com/in/masayukig/}{LinkedIn}}
      \item StoryBoad: \url{https://docs.openstack.org/infra/system-config/storyboard.html}
      \item Gerrit: \url{https://www.gerritcodereview.com/}
      \item Zuul: \url{https://zuul-ci.org/}
      \item Non-native English speakers in Open Source communities: \url{http://bit.ly/esl-yvr}
  \end{itemize}
\end{frame}

\end{document}
